\section{信息的存储}
\subsection{使用比特表示信息}
\subsubsection{万物皆比特}
由于信号易存储在双稳态单元中,并且可以在存在噪声和不准确的信道中可靠地传输。信息都可以使用二进制的编码进行表示,计算机通过二进制来发送指令,以及表示和处理各种数字、字符串等。

\subsubsection{字节数据编码}
1 Byte = 8 bits。二进制表示范围是 \(00000000_{2}\) 到 \(11111111_{2}\);十进制表示范围是 \(0_{10}\) 到 \(255_{10}\);十六进制表示范围是 \(00_{16}\) 到 \(FF_{16}\),十六进制以 16 为基数,计数符号为 0 - 9 和 A - F 。在 C 语言中,FA1D37B$_{16}$ 可表示为 0xFA1D37B 或 0xfa1d37b。
\begin{table}[H]
    \captionsetup{skip=4pt}
    \centering
    \setlength{\arrayrulewidth}{1pt}
    \begin{tabular}{ccc}
        \hline
        \makebox[0.2\textwidth][c]{十进制} & \makebox[0.2\textwidth][c]{十六进制} & \makebox[0.2\textwidth][c]{二进制} \\
        \noalign{\global\setlength{\arrayrulewidth}{0.5pt}}
        \hline
        0                               & 00                               & 00000000                        \\
        1                               & 01                               & 00000001                        \\
        2                               & 02                               & 00000010                        \\
        3                               & 03                               & 00000011                        \\
        4                               & 04                               & 00000100                        \\
        5                               & 05                               & 00000101                        \\
        6                               & 06                               & 00000110                        \\
        7                               & 07                               & 00000111                        \\
        8                               & 08                               & 00001000                        \\
        9                               & 09                               & 00001001                        \\
        10                              & 0A                               & 00001010                        \\
        11                              & 0B                               & 00001011                        \\
        12                              & 0C                               & 00001100                        \\
        13                              & 0D                               & 00001101                        \\
        14                              & 0E                               & 00001110                        \\
        15                              & 0F                               & 00001111                        \\
        \noalign{\global\setlength{\arrayrulewidth}{1pt}}
        \hline
    \end{tabular}
    \caption{字节数据编码对应表}
\end{table}
\subsection{位运算}
\subsubsection{布尔代数}
\begin{itemize}
    \item And(与):\(A \& B = 1\) 当且仅当 \(A = 1\) 且 \(B = 1\)。
    \item Or(或):\(A | B = 1\) 当 \(A = 1\) 或者 \(B = 1\)。
    \item Not(非):\(\sim A = 1\) 当 \(A = 0\)。
    \item Exclusive - Or(Xor,异或):\(A\^{}B = 1\) 当 \(A = 1\) 或者 \(B = 1\),但不同时为 1,即相同为 0,不同为 1。
\end{itemize}

\subsubsection{C 语言中的位运算}
C 语言定义了四个位运算符号:
\begin{table}[H]
    \captionsetup{skip=4pt}
    \centering
    \setlength{\arrayrulewidth}{1pt}
    \begin{tabular}{cccc}
        \hline
        \makebox[0.15\textwidth][c]{C 表达式} & \makebox[0.2\textwidth][c]{二进制表达式} & \makebox[0.2\textwidth][c]{二进制结果} & \makebox[0.15\textwidth][c]{十六进制结果} \\
        \noalign{\global\setlength{\arrayrulewidth}{0.5pt}}
        \hline
        $\sim$ 0x41                        & $\sim$[0100 0001]                  & \([10111110]\)                    & 0xBE                                \\
        $\sim $ 0x00                       & \(\sim[0000 0000]\)                & \([11111111]\)                    & 0xFF                                \\
        0x69\&0x55                         & \([0110 1001]\&[0101 0101]\)       & \([0100 0001]\)                   & 0x41                                \\
        0x69 | 0x55                        & \([0110 1001] | [0101 0101]\)      & \([01111101]\)                    & 0x7D                                \\
        \noalign{\global\setlength{\arrayrulewidth}{1pt}}
        \hline
    \end{tabular}
    \caption{C 语言位运算示例}
\end{table}
\subsubsection{异或运算的应用:数据交换}
在 C 语言中,可以利用异或运算实现不使用额外变量交换两个数:
\begin{minted}{c}
void funny(int *x, int *y)
{
    *x = *x ^ *y; /* #1 */
    *y = *x ^ *y; /* #2 */
    *x = *x ^ *y; /* #3 */
}
\end{minted}
交换过程如下:
\begin{table}[H]
    \captionsetup{skip=4pt}
    \centering
    \setlength{\arrayrulewidth}{1pt}
    \begin{tabular}{ccc}
        \hline
        \makebox[0.1\textwidth][c]{步骤} & \makebox[0.2\textwidth][c]{*x}                     & \makebox[0.2\textwidth][c]{*y}     \\
        \noalign{\global\setlength{\arrayrulewidth}{0.5pt}}
        \hline
        开始                             & A                                                  & B                                  \\
        1                              & A\^{}B                                             & B                                  \\
        2                              & A\^{}B                                             & (A\^{}B)\^{}B = A\^{}(B\^{}B) =  A \\
        3                              & (A\^{}B)\^{}A = (B\^{}A)\^{}A = B\^{}(A\^{}A)  = B & A                                  \\
        结束                             & B                                                  & A                                  \\
        \noalign{\global\setlength{\arrayrulewidth}{1pt}}
        \hline
    \end{tabular}
    \caption{异或运算交换数据过程}
\end{table}
\subsubsection{C 语言中的逻辑运算}
C 语言定义了三种逻辑运算:\(||\)(逻辑或)、\&\&(逻辑与)、\(!\)(逻辑非),具有短路效应。例如:
\begin{itemize}
    \item x \&\& 5/x 可以用于避免除 0 运算。
    \item p \&\& *p++ 可以避免空指针运算。
    \item 5 || x = y 赋值语句将不会被执行。
\end{itemize}
\subsubsection{C 语言中的移位运算}
C 语言中有逻辑移位和算术移位。右移运算有逻辑移位(左侧补 0)和算术移位(左侧补原最高位值)两种操作。对于无符号数,右移是逻辑的;对于有符号数,几乎所有的编译器针对有符号数的右移都采用的是算术右移
\begin{figure}[H]
    \centering
    \captionsetup{skip=4pt}
    \includegraphics[width=6cm]{3.png}
    \caption{移位运算}
\end{figure}
\subsubsection{未定义行为}
C 语言规范中没有被明确定义的行为称为未定义行为(UB),编程时应避免使用未定义行为,但有符号数算术右移除外。例如,移位 \(k\),当 \(k\) 大于等于变量位长时,值直接变为0,在 GCC 中的实现如下:
\begin{minted}{c}
int aval = 0x0EDCBA98 >> 36; 
movl $0, -8(%ebp)   // 值直接变为0
unsigned uval = 0xFEDCBA98u << 40;
movl $0, -4(%ebp)   // 值直接变为0
\end{minted}
\subsubsection{运算优先级}
移位运算符的优先级低于加减乘除,例如 \(- 1<<2+3<<4\),正确的运算顺序为 \((1<<(2 + 3))<<4\) 。
\subsection{信息的存储和表示}
\subsubsection{字长}
字长是指针数据的大小(虚拟地址宽度)。 32 位 ( 4 字节) 计算机字长限制了地址空间为 4GiB(\(2^{32}\) 字节);64 位字长( 8 字节)寻址能力达到了 18EiB
\subsubsection{C 语言中的各数据类型位宽}
计算机和编译器支持多种数据类型,或是小于字长,或大于字长,但长度都是整数个字节。
\begin{table}[H]
    \captionsetup{skip=4pt}
    \centering
    \setlength{\arrayrulewidth}{1pt}
    \begin{tabular}{cccc}
        \hline
        \makebox[0.15\textwidth][c]{C 语言数据类型} & \makebox[0.15\textwidth][c]{典型 32 位系统} & \makebox[0.15\textwidth][c]{典型 64 位系统} & \makebox[0.15\textwidth][c]{x86 - 64} \\
        \noalign{\global\setlength{\arrayrulewidth}{0.5pt}}
        \hline
        char                                  & 1                                      & 1                                      & 1                                     \\
        short                                 & 2                                      & 2                                      & 2                                     \\
        int                                   & 4                                      & 4                                      & 4                                     \\
        long                                  & 4                                      & 8                                      & 8                                     \\
        float                                 & 4                                      & 4                                      & 4                                     \\
        double                                & 8                                      & 8                                      & 8                                     \\
        pointer                               & 4                                      & 8                                      & 8                                     \\
        \noalign{\global\setlength{\arrayrulewidth}{1pt}}
        \hline
    \end{tabular}
    \caption{C 语言数据类型位宽}
\end{table}
\subsubsection{字节序}
有小端序(Little endian)和大端序(Big endian)两种。小端序如 Intel,低地址存放低位数据,高地址存放高位数据;大端序如 IBM、Sun Microsystem(Oracle),低地址存放高位数据,高地址存放低位数据。例如,对于 0x1234567 :
\begin{table}[H]
    \captionsetup{skip=4pt}
    \centering
    \setlength{\arrayrulewidth}{1pt}
    \begin{tabular}{ccccc}
        \hline
        \multicolumn{5}{|c|}{地址}                                                                                                                                       \\
        \cline{1-5}
        \makebox[0.1\textwidth][c]{} & \makebox[0.1\textwidth][c]{低地址} & \makebox[0.1\textwidth][c]{} & \makebox[0.1\textwidth][c]{} & \makebox[0.1\textwidth][c]{高地址} \\
        \noalign{\global\setlength{\arrayrulewidth}{0.5pt}}
        \hline
        大端序                          & \(0x100: 0x12\)                 & \(0x101: 0x34\)              & \(0x102: 0x56\)              & \(0x103: 0x07\)                 \\
        小端序                          & \(0x100: 0x07\)                 & \(0x101: 0x56\)              & \(0x102: 0x34\)              & \(0x103: 0x12\)                 \\
        \noalign{\global\setlength{\arrayrulewidth}{1pt}}
        \hline
    \end{tabular}
    \caption{字节序示例}
\end{table}
\subsubsection{探索数据在存储器中的存储方式}
通过以下代码可以打印各变量的字节表示形式:
\begin{minted}{c}
#include <stdio.h>
typedef unsigned char *byte_pointer;
void show_bytes(byte_pointer start,int len)
{
    int i;
    for(i = 0; i < len; i++)
        printf("%.2x ",start[i]);
    printf("\n");
}
void show_int(int x)
{
    show_bytes((byte_pointer)&x, sizeof(int));
}
void show_float(float x)
{
    show_bytes((byte_pointer)&x, sizeof(float));
}
void show_pointer(void *x)
{
    show_bytes((byte_pointer)x, sizeof(void*));
}
\end{minted}
在 Linux32/64(小端)、Win32(小端)和 Sun(32 位,大端)系统下的测试结果如下:
\begin{table}[H]
    \captionsetup{skip=4pt}
    \centering
    \setlength{\arrayrulewidth}{1pt}
    \begin{tabular}{cccc}
        \hline
        \makebox[0.1\textwidth][c]{机器} & \makebox[0.1\textwidth][c]{值}           & \makebox[0.1\textwidth][c]{类型} & \makebox[0.2\textwidth][c]{字节(十六进制)} \\
        \noalign{\global\setlength{\arrayrulewidth}{0.5pt}}
        \hline
        Linux 32                       & 12345                                   & int                            & 39 30 00 00                          \\
        Windows                        & 12345                                   & int                            & 39 30 00 00                          \\
        Sun                            & 12345                                   & int                            & 00 00 30 39                          \\
        Linux 64                       & 12345                                   & int                            & 39 30 00 00                          \\
        Linux 32                       & 12345.0                                 & float                          & 00 e4 40 46                          \\
        Windows                        & 12345.0                                 & float                          & 00 e4 40 46                          \\
        Sun                            & 12345.0                                 & float                          & 46 40 e4 00                          \\
        Linux 64                       & 12345.0                                 & float                          & 00 e4 40 46                          \\
        Linux 32                       & \(      \& ival\)                       & int *                          & e4 f9 ff bf                          \\
        Windows                        & \(                            \& ival\) & int *                          & b4 cc 22 00                          \\
        Sun                            & \(                            \& ival\) & int *                          & ef ff fa 0c                          \\
        Linux 64                       & \(                            \& ival\) & int *                          & b8 11 e5 ff ff 7f 00 00              \\
        \noalign{\global\setlength{\arrayrulewidth}{1pt}}
        \hline
    \end{tabular}
    \caption{不同系统下数据存储测试结果}
\end{table}
\subsubsection{指针的存储方法}
不同的编译器和计算机可能会分配不同的地址,甚至每一次运行时得到的结果都不相同。例如,在 x86 - 64、Sun、IA32 环境下:
\begin{minted}{c}
int B = -15213; 
int *P = &B;
\end{minted}
\subsubsection{字符串的表示}
C 语言的字符串使用 char 数组表示,每个字符都被编码成 ASCII 码,是一个 7 比特的字符编码集(扩展集为 8 比特)。字符“0”的编码是 \(0x30\),数字字符 \(i\) 的编码是 \(0x30 + i\) 。字符串的结尾应为空字符,即 ASCII 编码为 0 。字符串的表示与字节序无关,大小端兼容。
\subsubsection{程序的表示}
不同类型的机器使用不同的且不兼容的指令和指令编码。在相同处理器不同的操作系统中,由于编码规范存在差异,同样代码所生成的程序也不是二进制兼容的,程序很少能够在不同类型机器和不同操作系统中实现二进制水平上移植。
\subsubsection{小知识:PE 和 ELF 格式}
Windows 操作系统下常用的可执行文件格式是 PE(Portable Executable);Unix 家族(含 Linux)操作系统下可执行文件格式为 ELF(Executable and Linkable Format)。
