\section{整数}
\subsection{编码}
\subsubsection{整数的表示}
不同C语言数据类型在32位和64位机器上有不同的取值范围:
\begin{table}[H]
    \captionsetup{skip=4pt}
    \centering
    \setlength{\arrayrulewidth}{1pt}
    \begin{tabular}{ccc}
        \hline
        \makebox[0.2\textwidth][c]{C语言数据类型} & \makebox[0.2\textwidth][c]{32位机器} & \makebox[0.2\textwidth][c]{64位机器}             \\
        \noalign{\global\setlength{\arrayrulewidth}{0.5pt}}
        \hline
        char                                & $-128\sim 127$                    & $-128\sim127$                                 \\
        unsigned char                       & $0\sim255$                        & $0\sim255$                                    \\
        short                               & $-32768\sim32767$                 & $-32768\sim32767$                             \\
        unsigned short                      & $0\sim65535$                      & $0\sim65535$                                  \\
        int                                 & $-2147483648\sim2147483647$       & $-2147483648\sim2147483647$                   \\
        unsigned int                        & $0\sim4294967295 $                & $0\sim4294967295$                             \\
        long                                & $-2147483648\sim2147483647$       & $-9223372036854775808\sim9223372036854775807$ \\
        unsigned long                       & $0\sim4294967295$                 & $0\sim18446744073709551615$                   \\

        \noalign{\global\setlength{\arrayrulewidth}{1pt}}
        \hline
    \end{tabular}
    \caption{C语言数据类型的取值范围}
\end{table}
\subsubsection{二进制转无符号数和有符号数(补码)}
假设整数数据类型有\(w\)位,二进制编码用位向量\(\vec{x}=(x_{w - 1},x_{w - 2},\cdots,x_{0})\)表示。
无符号数编码定义为:\(B2U_{w}(\vec{x})=\sum\limits_{i = 0}^{w - 1}x_{i}2^{i}\)
有符号数(补码)编码定义为:\(B2T_{w}(\vec{x})=-x_{w - 1}2^{w - 1}+\sum\limits_{i = 0}^{w - 2}x_{i}2^{i}\)

例如:
\[B2U_{4}([0001])=0\times2^{3}+0\times2^{2}+0\times2^{1}+1\times2^{0}=1\]
\[B2U_{4}([1011])=1\times2^{3}+0\times2^{2}+1\times2^{1}+1\times2^{0}=11\]

\[B2T_{4}([0001])=-0\times2^{3}+0\times2^{2}+0\times2^{1}+1\times2^{0}=1\]
\[B2T_{4}([1011])=-1\times2^{3}+0\times2^{2}+1\times2^{1}+1\times2^{0}=-5\]

\subsubsection{将整数转换为补码}
如果数字值非负,补码等于该值对应的二进制数(位长不足补0);如果数字值为负,其绝对值的二进制数逐位取反,并加1,符号位置1 。例如,假设整数类型位长为4,整数5的补码是0101,整数 -5的补码是1011。
\subsubsection{可表示的整数范围}
无符号数范围:\(U_{min}=0\),\(U_{max}=2^{w}-1\)

有符号数范围:\(T_{min}=-2^{w - 1}\),\(T_{max}=2^{w - 1}-1\)

关系满足\(\vert T_{Min}\vert=T_{Max}+1\),\(U_{max}=2\times T_{Max}+1\)。例如:
\begin{table}[H]
    \captionsetup{skip=4pt}
    \centering
    \setlength{\arrayrulewidth}{1pt}
    \begin{tabular}{cccc}
        \hline
        \makebox[0.1\textwidth][c]{Value} & \makebox[0.08\textwidth][c]{Word size \(w = 8\)} & \makebox[0.08\textwidth][c]{\(w = 16\)} & \makebox[0.08\textwidth][c]{\(w = 32\)} \\
        \noalign{\global\setlength{\arrayrulewidth}{0.5pt}}
        \hline
        $U_{Max}$                              & 0xFF (255)                                       & 0xFFFF (65535)                                    & 0xFFFFFFFF (4294967295)                           \\
        $T_{Min}$                              & 0x80 (-128)                                      & 0x8000 (-32768)                                   & 0x80000000 (-2147483648)                          \\
        $T_{Max}$                              & 0x7F (127)                                       & 0x7FFF (32767)                                    & 0x7FFFFFFF (2147483647)                           \\
        -1                                & 0xFF                                             & 0xFFFF                                            & 0xFFFFFFFF                                        \\
        0                                 & 0x00                                             & 0x0000                                            & 0x00000000                                        \\
        \noalign{\global\setlength{\arrayrulewidth}{1pt}}
        \hline
    \end{tabular}
    \caption{不同字长下的重要数字}
\end{table}

\subsubsection{反码和原码}
反码:是符号位的权重为\(-(2^{w - 1}-1)\)。

原码:符号位的作用仅用于决定正/负,1负2正。
\subsection{变换}
\subsubsection{有符号数转无符号数}
C语言允许将有符号数转换为无符号数,转换时编码本身没有发生变化。非负数转换后值不变,负数转换为一个大整数。例如:
\[ux=\begin{cases}x, & x\geq0\\x + 2^{w}, & x<0\end{cases}\]
\subsubsection{无符号数转有符号数}
无符号数转有符号数时编码保持不变,转换规则为:
\[x=\begin{cases}ux, & ux\leq T_{Max}\\ux - 2^{w}, & ux>T_{Max}\end{cases}\]
\subsubsection{C语言中的有符号数和无符号数}
缺省情况下,所有的整数常量都是有符号数。如果需要声明无符号数常量,需要增加一个后缀“U”,如0U,4294967259U。C语言中有显式转换和隐式转换。隐式转换时,编译器会产生Warning,不推荐使用。例如:
\begin{minted}{c}
int tx, ty;
unsigned ux, uy;
tx = (int)ux; // 显式转换
uy = (unsigned)ty; // 显式转换
tx = ux; // 隐式转换
uy = ty; // 隐式转换
\end{minted}
\subsubsection{C语言中的表达式求值}
如果在一个表达式中混用无符号数和有符号数,有符号数会被隐式转换成无符号数,比较运算也采用此规则。例如:
\begin{table}[H]
    \captionsetup{skip=4pt}
    \centering
    \setlength{\arrayrulewidth}{1pt}
    \begin{tabular}{cccc}
        \hline
        \makebox[0.1\textwidth][c]{Constant1} & \makebox[0.08\textwidth][c]{Relation} & \makebox[0.1\textwidth][c]{Constant2} & \makebox[0.1\textwidth][c]{Evaluation} \\
        \noalign{\global\setlength{\arrayrulewidth}{0.5pt}}
        \hline
        0                                     & ==                                    & 0U                                    & unsigned                               \\
        -1                                    & <                                     & 0                                     & signed                                 \\
        -1                                    & >                                     & 0U                                    & unsigned                               \\
        2147483647                            & >                                     & -2147483648                           & signed                                 \\
        2147483647U                           & <                                     & -2147483648                           & unsigned                               \\
        -1                                    & >                                     & -2                                    & signed                                 \\
        (unsigned)-1                          & >                                     & -2                                    & unsigned                               \\
        \noalign{\global\setlength{\arrayrulewidth}{1pt}}
        \hline
    \end{tabular}
    \caption{表达式求值示例(以32位字长为例)}
\end{table}
\subsubsection{位扩展}
无符号数位扩展是零扩展,扩展后最高位的空位补0;

有符号数位扩展是符号位扩展,扩展后最高位的空位用原编码的符号位填充。例如:
\begin{table}[H]
    \captionsetup{skip=4pt}
    \centering
    \setlength{\arrayrulewidth}{1pt}
    \begin{tabular}{cccc}
        \hline
        \makebox[0.08\textwidth][c]{变量} & \makebox[0.08\textwidth][c]{Decimal} & \makebox[0.08\textwidth][c]{Hex} & \makebox[0.15\textwidth][c]{Binary} \\
        \noalign{\global\setlength{\arrayrulewidth}{0.5pt}}
        \hline
        short int x                     & 12345                                & 30 39                            & 00110000 00111001                   \\
        int ix                          & 12345                                & 00 00 30 39                      & 00000000 00000000110000 00111001    \\
        short int y                     & -12345                               & CF C7                            & 11001111 11000111                   \\
        int iy                          & -12345                               & FF FF CF C7                      & 11111111 11111111 11001111 11000111 \\
        \noalign{\global\setlength{\arrayrulewidth}{1pt}}
        \hline
    \end{tabular}
    \caption{位扩展示例}
\end{table}
\subsubsection{位截断}
无符号数和有符号数的位截断都是将高位的位直接丢弃,只保留低位的位。无符号数截断后的值为 \(x\bmod 2^{k}\)(\(k\) 为截断后保留的位数);有符号数截断后的值为将截断后的无符号数转换为有符号数。例如:
\begin{minted}{c}
int x = 53191;
short sx = (short)x; // 截断
int y = sx; // 再扩展
\end{minted}
这里 \(x = 53191\),二进制为 \(0000 0000 0000 0000 1100 1111 1100 0111\),截断为 \(16\) 位后 \(sx\) 二进制为 \(1100 1111 1100 0111\),再扩展为 \(32\) 位 \(y\) 二进制为 \(1111 1111 1111 1111 1100 1111 1100 0111\),值为 \(-12345\)。
\subsection{运算}
\subsubsection{无符号数加法}
对于无符号数 \(x\) 和 \(y\),如果 \(x + y<2^{w}\),则正常相加;如果 \(x + y\geq2^{w}\),则结果为 \((x + y)\bmod 2^{w}\),即产生了溢出。例如:
\begin{table}[H]
    \captionsetup{skip=4pt}
    \centering
    \setlength{\arrayrulewidth}{1pt}
    \begin{tabular}{cccc}
        \hline
        \makebox[0.1\textwidth][c]{x} & \makebox[0.1\textwidth][c]{y} & \makebox[0.1\textwidth][c]{$x + y$} & \makebox[0.1\textwidth][c]{$x + y\bmod 2^{4}$} \\
        \noalign{\global\setlength{\arrayrulewidth}{0.5pt}}
        \hline
        0                             & 0                             & 0                                   & 0                                              \\
        5                             & 3                             & 8                                   & 8                                              \\
        11                            & 13                            & 24                                  & 8                                              \\
        15                            & 15                            & 30                                  & 14                                             \\
        \noalign{\global\setlength{\arrayrulewidth}{1pt}}
        \hline
    \end{tabular}
    \caption{无符号数加法示例(\(w = 4\))}
\end{table}
\subsubsection{有符号数加法}
对于有符号数 \(x\) 和 \(y\),可能会出现正溢出、正常和负溢出三种情况。
\begin{enumerate}
    \item 正溢出时 \(x + y\geq2^{w - 1}\),结果为 \(x + y-2^{w}\);
    \item 正常时 \(-2^{w - 1}\leq x + y<2^{w - 1}\),结果为 \(x + y\);
    \item 负溢出时 \(x + y<-2^{w - 1}\),结果为 \(x + y + 2^{w}\)。
\end{enumerate}


例如:
\begin{table}[H]
    \captionsetup{skip=4pt}
    \centering
    \setlength{\arrayrulewidth}{1pt}
    \begin{tabular}{cccc}
        \hline
        \makebox[0.1\textwidth][c]{x} & \makebox[0.1\textwidth][c]{y} & \makebox[0.1\textwidth][c]{x + y} & \makebox[0.1\textwidth][c]{补码加法结果} \\
        \noalign{\global\setlength{\arrayrulewidth}{0.5pt}}
        \hline
        1                             & 2                             & 3                                 & 3                                  \\
        5                             & 5                             & 10                                & -6                                 \\
        -5                            & -5                            & -10                               & 6                                  \\
        -1                            & -2                            & -3                                & -3                                 \\
        \noalign{\global\setlength{\arrayrulewidth}{1pt}}
        \hline
    \end{tabular}
    \caption{有符号数加法示例(\(w = 4\))}
\end{table}

\subsubsection{无符号数乘法}
无符号数 \(x\) 和 \(y\) 相乘,结果为 \((x\times y)\bmod 2^{w}\)。例如在 \(w = 4\) 时,\(x = 3,y = 5\),\(x\times y = 15\),结果就是 \(15\);若 \(x = 5,y = 6\),\(x\times y = 30\),结果为 \(30\bmod 16 = 14\)。
\subsubsection{有符号数乘法}
有符号数 \(x\) 和 \(y\) 相乘,先按无符号数相乘得到结果 \(p=x\times y\),然后将 \(p\) 截断为 \(w\) 位,再将截断后的无符号数转换为有符号数。例如在 \(w = 4\) 时,\(x=-3,y = 2\),无符号相乘 \(x\) 看作 \(13\),\(y = 2\),\(p = 26\),截断为 \(4\) 位后为 \(10\),转换为有符号数为 \(-6\)。
\subsubsection{移位运算}
在大多数计算机上移位运算和加法运算比乘法运算快得多。左移 \(k\) 位相当于乘以 \(2^{k}\)。对于无符号数右移 \(k\) 位使用逻辑右移,相当于除以 \(2^{k}\),采用的是向下取整;对于有符号数右移 \(k\) 位,采用算术右移,相当于除以 \(2^{k}\) 并向下取整。例如:
\begin{minted}{c}
int x = 12;
int y = x << 2; // y = 12 * 4 = 48
int z = x >> 2; // z = 12 / 4 = 3
\end{minted}
对于有符号数除以 \(2^{k}\),为了实现向零取整,可以在右移前加上一个偏置量 \((1<<k)-1\)。相当于右移$k$位后再加1。代码实现如下:
\begin{minted}{c}
int divide_power2(int x, int k) {
    int bias = (x < 0)? ((1 << k) - 1) : 0;
    return (x + bias) >> k;
}
\end{minted}