\section{程序的机器级表示:基本操作}
\subsection{x86寄存器}
\subsubsection{x86 - 64寄存器}
x86 - 64架构下,每个寄存器的低4/2/1字节都有唯一的标识。其包含多个寄存器,如 \mintinline{asm}{%rax}、\mintinline{asm}{%rbx}、\mintinline{asm}{%rcx} 等,可以用于存储计算结果、函数参数传递、循环计数等。其中 \mintinline{asm}{%rsp} 是栈指针寄存器,有使用限制。寄存器具体如下:

\begin{table}[H]
    \captionsetup{skip=4pt}
    \centering
    \setlength{\arrayrulewidth}{1pt}
    \begin{tabular}{cccc}
        \hline
        63 - 31位 & 31 - 15位 & 15 - 7位 & 7 - 0位 \\
        \hline
        \mintinline{asm}{%rax} & \mintinline{asm}{%eax}  & \mintinline{asm}{%ax}   & \mintinline{asm}{%al}   \\
        \mintinline{asm}{%rbx} & \mintinline{asm}{%ebx}  & \mintinline{asm}{%bx}   & \mintinline{asm}{%bl}   \\
        \mintinline{asm}{%rcx} & \mintinline{asm}{%ecx}  & \mintinline{asm}{%cx}   & \mintinline{asm}{%cl}   \\
        \mintinline{asm}{%rdx} & \mintinline{asm}{%edx}  & \mintinline{asm}{%dx}   & \mintinline{asm}{%dl}   \\
        \mintinline{asm}{%rsi} & \mintinline{asm}{%esi}  & \mintinline{asm}{%si}   & \mintinline{asm}{%sil}  \\
        \mintinline{asm}{%rdi} & \mintinline{asm}{%edi}  & \mintinline{asm}{%di}   & \mintinline{asm}{%dil}  \\
        \mintinline{asm}{%rbp} & \mintinline{asm}{%ebp}  & \mintinline{asm}{%bp}   & \mintinline{asm}{%bp}   \\
        \mintinline{asm}{%rsp} & \mintinline{asm}{%esp}  & \mintinline{asm}{%sp}   & \mintinline{asm}{%spl}  \\
        \mintinline{asm}{%r8}  & \mintinline{asm}{%r8d}  & \mintinline{asm}{%r8w}  & \mintinline{asm}{%r8b}  \\
        \mintinline{asm}{%r9}  & \mintinline{asm}{%r9d}  & \mintinline{asm}{%r9w}  & \mintinline{asm}{%r9b}  \\
        \mintinline{asm}{%r10} & \mintinline{asm}{%r10d} & \mintinline{asm}{%r10w} & \mintinline{asm}{%r10b} \\
        \mintinline{asm}{%r11} & \mintinline{asm}{%r11d} & \mintinline{asm}{%r11w} & \mintinline{asm}{%r11b} \\
        \mintinline{asm}{%r12} & \mintinline{asm}{%r12d} & \mintinline{asm}{%r12w} & \mintinline{asm}{%r12b} \\
        \mintinline{asm}{%r13} & \mintinline{asm}{%r13d} & \mintinline{asm}{%r13w} & \mintinline{asm}{%r13b} \\
        \mintinline{asm}{%r14} & \mintinline{asm}{%r14d} & \mintinline{asm}{%r14w} & \mintinline{asm}{%r14b} \\
        \mintinline{asm}{%r15} & \mintinline{asm}{%r15d} & \mintinline{asm}{%r15w} & \mintinline{asm}{%r15b} \\
        \hline
    \end{tabular}
    \caption{x86 - 64寄存器各部分标识}
\end{table}
\subsubsection{IA32 (x86 - 32)寄存器}
IA32(x86 - 32)架构下的寄存器有 \mintinline{asm}{%eax}、\mintinline{asm}{%ecx}、\mintinline{asm}{%edx} 等,其各部分标识如下:
\begin{table}[H]
    \captionsetup{skip=4pt}
    \centering
    \setlength{\arrayrulewidth}{1pt}
    \begin{tabular}{cccc}
        \hline
        31 - 15位 & 15 - 7位 & 7 - 0位 \\
        \hline
        \mintinline{asm}{%eax} & \mintinline{asm}{%ax} & \mintinline{asm}{%al} \\
        \mintinline{asm}{%ecx} & \mintinline{asm}{%cx} & \mintinline{asm}{%cl} \\
        \mintinline{asm}{%edx} & \mintinline{asm}{%dx} & \mintinline{asm}{%dl} \\
        \mintinline{asm}{%ebx} & \mintinline{asm}{%bx} & \mintinline{asm}{%bl} \\
        \mintinline{asm}{%esi} & \mintinline{asm}{%si} & -                      \\
        \mintinline{asm}{%edi} & \mintinline{asm}{%di} & -                      \\
        \mintinline{asm}{%esp} & \mintinline{asm}{%sp} & -                      \\
        \mintinline{asm}{%ebp} & \mintinline{asm}{%bp} & -                      \\
        \hline
    \end{tabular}
    \caption{IA32 (x86 - 32)寄存器各部分标识}
\end{table}
\subsection{数据移动指令}
\subsubsection{汇编语言格式}
汇编语言格式为 \mintinline{asm}{[label :] [opcode] [operand 1] [, operand 2]},即\[\text{标号:}\quad \text{操作码}\quad \text{操作数}1\quad [\text{,操作数}2]\]不同风格的汇编语句结构有差异。例如,在ATT assembly中:
\begin{minted}{asm}
l1: movq $5, %rax
addq $-16, (%rax)
\end{minted}
在Intel assembly中:
\begin{minted}{asm}
l1: mov rax, 5
add QWORD PTR[rax], -16
\end{minted}
\subsubsection{操作数类型}
\begin{enumerate}[label=\arabic*.]
    \item \textbf{立即数}:整数常量,如 \mintinline{asm}{$0x400}、\mintinline{asm}{$-533} ,和C语言中的常数类似,但需要加前缀 \mintinline{asm}{$},其被编码为1、2、4或8个字节。
    \item \textbf{寄存器}:十六个整数寄存器之一,如 \mintinline{asm}{%rax}、\mintinline{asm}{%r13} 。其中 \mintinline{asm}{%rsp} 有特殊用途,通常不使用,其他寄存器在一些特殊的指令中也会有特殊用途。
    \item \textbf{存储器}:指向的内存中8个连续字节,由寄存器给出地址,例如 \mintinline{asm}{(%rax)} ,还有很多其他的“寻址模式”。
\end{enumerate}
\subsubsection{movq指令操作数的几种组合}
\begin{table}[H]
    \captionsetup{skip=4pt}
    \centering
    \setlength{\arrayrulewidth}{1pt}
    \begin{tabular}{cccc}
        \hline
        源操作数Src & 目标操作数Dest & 示例                             & 等价C语言 \\
        \hline
        立即数     & 寄存器       & \mintinline{asm}{movq $0x4,            %rax}    & \mintinline{c}{temp = 0x4;}    \\
        立即数     & 存储器       & \mintinline{asm}{movq $-147, (         %rax)} & \mintinline{c}{*p = -147;}     \\
        寄存器     & 寄存器       & \mintinline{asm}{movq                  %rax, %rdx}    & \mintinline{c}{temp2 = temp1;} \\
        寄存器     & 存储器       & \mintinline{asm}{movq                  %rax, (%rdx)}  & \mintinline{c}{*p = temp;}     \\
        存储器     & 寄存器       & \mintinline{asm}{movq (                %rax), %rdx}  & \mintinline{c}{temp = *p;}     \\
        \hline
    \end{tabular}
    \caption{movq指令操作数组合及等价C语言表达}
\end{table}
\subsubsection{数据格式}
不同C语言类型声明对应不同的数据类型、操作码后缀和大小,具体如下:
\begin{table}[H]
    \captionsetup{skip=4pt}
    \centering
    \setlength{\arrayrulewidth}{1pt}
    \begin{tabular}{cccc}
        \hline
        C语言类型声明                & 数据类型             & 操作码后缀 & 大小(bytes) \\
        \hline
        \mintinline{c}{char}   & Byte             & b     & 1         \\
        \mintinline{c}{short}  & Word             & w     & 2         \\
        \mintinline{c}{int}    & Double Word      & l     & 4         \\
        \mintinline{c}{long}   & Quad Word        & q     & 8         \\
        \mintinline{c}{char *} & Quad Word        & q     & 8         \\
        \mintinline{c}{float}  & Single precision & s     & 4         \\
        \mintinline{c}{double} & Double precision & l     & 8         \\
        \hline
    \end{tabular}
    \caption{C语言类型与数据格式对应关系}
\end{table}
\subsubsection{几种简单的存储器寻址模式}
\begin{enumerate}[label=\arabic*.]
    \item \textbf{间接寻址}: \mintinline{asm}{(R)} 表示 \mintinline{asm}{Mem[Reg[R]]} ,寄存器 \mintinline{asm}{R} 指向了存储器的地址,和C语言中的指针作用相同,例如 \mintinline{asm}{movq (%rcx),%rax} 。
    \item \textbf{基地址 + 偏移量寻址}: \mintinline{asm}{D(R)} 表示 \mintinline{asm}{Mem[Reg[R]+D]} ,寄存器 \mintinline{asm}{R} 指定了存储器区域的开始位置,常数 \mintinline{asm}{D} 是偏移量,例如 \mintinline{asm}{movq 8(%rbp),%rdx} 。
\end{enumerate}
\subsubsection{举例:简单寻址模式(swap函数)}
以 \mintinline{c}{swap} 函数为例:
\begin{minted}{c}
void swap(long *xp, long *yp) {
    long t0 = *xp;
    long t1 = *yp; 
    *xp = t1; 
    *yp = t0;
}
\end{minted}
其汇编代码如下:
\begin{minted}{asm}
swap:
    movq (%rdi), %rax       # t0 = *xp
    movq (%rsi), %rdx       # t1 = *yp
    movq %rdx, (%rdi)       # *xp = t1
    movq %rax, (%rsi)       # *yp = t0
    ret
\end{minted}
在该函数中, \mintinline{asm}{%rdi} 存放 \mintinline{c}{xp} 的值, \mintinline{asm}{%rsi} 存放 \mintinline{c}{yp} 的值, \mintinline{asm}{%rax} 用于存放 \mintinline{c}{t0} , \mintinline{asm}{%rdx} 用于存放 \mintinline{c}{t1} 。通过逐步执行这些指令,实现两个指针所指向的值的交换。
\subsubsection{完整的存储器寻址模式}
最通用的形式为 \mintinline{asm}{D(Rb, Ri, S)} 表示 \mintinline{asm}{Mem[Reg[Rb]+S * Reg[Ri]+D]} ,其中:
\begin{enumerate}[label=\arabic*.]
    \item \mintinline{asm}{D}:常数偏移量,可以为1、2、4或8字节整数。
    \item \mintinline{asm}{Rb}:基地址寄存器,是16个寄存器之一。
    \item \mintinline{asm}{Ri}:变址寄存器,除 \mintinline{asm}{%rsp} 外的其他寄存器。
    \item \mintinline{asm}{S}:比例因子,可以为1、2、4或8。
\end{enumerate}

还有一些特殊形式,如 \mintinline{asm}{(Rb, Ri)} 表示 \mintinline{asm}{Mem[Reg[Rb] + Reg[Ri]]} ,

\mintinline{asm}{D(Rb, Ri)} 表示 \mintinline{asm}{Mem[Reg[Rb] + Reg[Ri] + D]} , \mintinline{asm}{(Rb, Ri, S)} 表示 \mintinline{asm}{Mem[Reg[Rb] + S * Reg[Ri]]} 。
\subsubsection{小练习:地址计算}
假设 \mintinline{asm}{%rdx = 0xf000}, \mintinline{asm}{%rcx = 0x0100} ,不同地址表达式的计算结果如下:
\begin{table}[H]
    \captionsetup{skip=4pt}
    \centering
    \setlength{\arrayrulewidth}{1pt}
    \begin{tabular}{ccc}
        \hline
        表达式 & 地址计算 & 地址 \\
        \hline
        \mintinline{asm}{0x8(%rdx)}      & \mintinline{asm}{0xf000 + 0x8}     & \mintinline{asm}{0xf008}  \\
        \mintinline{asm}{(%rdx,%rcx)}   & \mintinline{asm}{0xf000 + 0x100}   & \mintinline{asm}{0xf100}  \\
        \mintinline{asm}{(%rdx,%rcx,4)} & \mintinline{asm}{0xf000 + 4*0x100} & \mintinline{asm}{0xf400}  \\
        \mintinline{asm}{0x80(,%rdx,2)}  & \mintinline{asm}{2*0xf000 + 0x80}  & \mintinline{asm}{0x1e080} \\
        \hline
    \end{tabular}
    \caption{不同地址表达式的计算结果}
\end{table}
\subsection{算术、逻辑运算指令}
\subsubsection{地址计算指令(leaq)}
\mintinline{asm}{leaq} 指令格式为 \mintinline{asm}{leaq Src, Dst},其中 \mintinline{asm}{Src} 是寻址模式表达式,该指令将表达式计算的地址写入 \mintinline{asm}{Dst} 。其用途包括计算地址(计算过程中不需要引用存储器),例如 \mintinline{c}{p = & x[i]} ;还可计算模式为 \mintinline{c}{x + k*y}( \mintinline{c}{k = 1, 2, 4} 或8)的表达式。例如,对于函数:
\begin{minted}{c}
long m12(long x)
{
    return x*12;
}
\end{minted}
编译后的汇编指令为:
\begin{minted}{asm}
leaq    (%rdi,%rdi,2), %rax    # t <- x+x*2
salq    $2, %rax               # return t<<2
\end{minted}
\subsubsection{一些算术运算指令(两操作数指令)}
两操作数指令(双目运算)的格式及计算方式如下:
\begin{table}[H]
    \captionsetup{skip=4pt}
    \centering
    \setlength{\arrayrulewidth}{1pt}
    \begin{tabular}{ccc}
        \hline
        格式                                & 计算                                   & 说明                            \\
        \hline
        \mintinline{asm}{addq Src, Dest}  & \mintinline{asm}{Dest = Dest + Src}  & -                             \\
        \mintinline{asm}{subq Src, Dest}  & \mintinline{asm}{Dest = Dest - Src}  & -                             \\
        \mintinline{asm}{imulq Src, Dest} & \mintinline{asm}{Dest = Dest * Src}  & -                             \\
        \mintinline{asm}{salq Src, Dest}  & \mintinline{asm}{Dest = Dest << Src} & 也叫 \mintinline{asm}{shlq}     \\
        \mintinline{asm}{sarq Src, Dest}  & \mintinline{asm}{Dest = Dest >> Src} & 算术右移                          \\
        \mintinline{asm}{shrq Src, Dest}  & \mintinline{asm}{Dest = Dest >> Src} & 逻辑右移                          \\
        \mintinline{asm}{xorq Src, Dest}  & \mintinline{asm}{Dest = Dest ^ Src}  & -                             \\
        \mintinline{asm}{andq Src, Dest}  & \mintinline{asm}{Dest = Dest         & Src}                      & - \\
        \mintinline{asm}{orq Src, Dest}   & \mintinline{asm}{Dest = Dest | Src}  & -                             \\
        \hline
    \end{tabular}
    \caption{两操作数算术运算指令}
\end{table}
需要注意操作数的顺序,并且有符号数和无符号数指令没有区别。
\subsubsection{一些算术运算指令(单操作数指令)}
单操作数指令(单目运算)的格式及计算方式如下:
\begin{table}[H]
    \captionsetup{skip=4pt}
    \centering
    \setlength{\arrayrulewidth}{1pt}
    \begin{tabular}{cc}
        \hline
        格式                          & 计算                                \\
        \hline
        \mintinline{asm}{incq Dest} & \mintinline{asm}{Dest = Dest + 1} \\
        \mintinline{asm}{decq Dest} & \mintinline{asm}{Dest = Dest - 1} \\
        \mintinline{asm}{negq Dest} & \mintinline{asm}{Dest = - Dest}   \\
        \mintinline{asm}{notq Dest} & \mintinline{asm}{Dest = ~Dest}    \\
        \hline
    \end{tabular}
    \caption{单操作数算术运算指令}
\end{table}
更多指令可查阅教材。
\subsubsection{需要关注的指令及举例:算术运算}
需要关注的指令有 \mintinline{asm}{leaq}(地址计算)、 \mintinline{asm}{salq}(左移)、 \mintinline{asm}{imulq}(乘法)等。以函数:
\begin{multicols}{2}
    \begin{minted}{c}
long arith(long x, long y, long z)
{
    long t1 = x+y;
    long t2 = z+t1;
    long t3 = x+4;
    long t4 = y * 48;
    long t5 = t3 + t4;
    long rval = t2 * t5;
    return rval;
}
\end{minted}
    \columnbreak
    \begin{minted}{asm}
arith:
    addq %rsi, %rdi       # t1 = x + y
    addq %rdi, %rdx       # t2 = z + t1
    leaq 4(%rdi), %rax    # t3 = x + 4
    leaq (%rsi,%rsi,2), %rcx  # t4a = y + 2*y
    salq $4, %rcx         # t4 = t4a << 4
    addq %rcx, %rax       # t5 = t3 + t4
    imulq %rax, %rdx      # rval = t2 * t5
    movq %rdx, %rax       # return rval
    ret
\end{minted}
\end{multicols}

% 此函数的汇编代码通过一系列指令实现了函数中各变量的计算。首先将 \mintinline{asm}{x} 和 \mintinline{asm}{y} 相加得到 \mintinline{asm}{t1},再把 \mintinline{asm}{t1} 与 \mintinline{asm}{z} 相加得到 \mintinline{asm}{t2}。对于 \mintinline{asm}{t3},使用 \mintinline{asm}{leaq} 指令计算 \mintinline{asm}{x + 4}。计算 \mintinline{asm}{t4} 时,先通过 \mintinline{asm}{leaq} 计算 \mintinline{asm}{y + 2*y} 得到 \mintinline{asm}{t4a},再左移4位得到 \mintinline{asm}{t4}。然后将 \mintinline{asm}{t3} 和 \mintinline{asm}{t4} 相加得到 \mintinline{asm}{t5},最后将 \mintinline{asm}{t2} 和 \mintinline{asm}{t5} 相乘得到结果 \mintinline{asm}{rval} 并返回。
