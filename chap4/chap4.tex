\section{程序的机器级表示:过程}

在计算机系统中,程序的机器级表示是理解程序如何在底层运行的关键。过程(函数调用)是程序设计中的重要概念,它涉及到控制流的转移、数据的传递以及栈的管理等多个方面。下面将详细介绍程序机器级表示中过程相关的知识。

\subsection{栈的结构}

\subsubsection{x86 - 64 的栈概述}

在 x86 - 64 架构的计算机系统中,栈是一种非常重要的数据结构,用于存储过程调用的上下文信息。栈的一个显著特点是它向低地址方向生长,这意味着当有新的数据入栈时,栈指针会向低地址移动。寄存器 \mintinline{asm}{%rsp} 用于存储栈的最低地址,也就是栈指针,它始终指向栈顶的位置。

栈的基本特性总结如下:
\begin{table}[H]
    \captionsetup{skip=4pt}
    \centering
    \setlength{\arrayrulewidth}{1pt}
    \begin{tabular}{c}
        \hline
        \makebox[0.8\textwidth][c]{栈的特性}             \\
        \noalign{\global\setlength{\arrayrulewidth}{0.5pt}}
        \hline
        栈底(Stack “Bottom”)位于高地址端,这是栈的起始位置。           \\
        栈顶(Stack “Top”)位于低地址端,随着数据的入栈和出栈,栈顶的位置会不断变化。 \\
        栈向下生长(Stack Grows Down),即向低地址方向扩展。当有新的数据需要入栈时,栈指针 \mintinline{asm}{%rsp} 会减小。 \\
        栈指针(Stack Pointer) \mintinline{asm}{%rsp} 始终指向栈顶,通过操作 \mintinline{asm}{%rsp} 可以实现数据的入栈和出栈操作。 \\
        \noalign{\global\setlength{\arrayrulewidth}{1pt}}
        \hline
    \end{tabular}
    \caption{x86 - 64 栈的基本特性}
\end{table}

\subsubsection{x86 - 64 的栈:入栈操作}

入栈操作 \mintinline{asm}{pushq src} 是将一个 64 位的数据从源操作数 \mintinline{asm}{src} 压入栈中。具体步骤如下:

\begin{enumerate}
    \item 从 \mintinline{asm}{src} 中取出操作数(Fetch operand at \mintinline{asm}{src}):首先,处理器会从源操作数 \mintinline{asm}{src} 所在的位置读取要入栈的数据。这个源操作数可以是寄存器或者内存地址。
    \item \mintinline{asm}{%rsp} 减 8(Decrement \mintinline{asm}{%rsp} by 8):由于在 x86 - 64 架构中,每个数据单元是 64 位(8 字节),所以在入栈时,栈指针 \mintinline{asm}{%rsp} 需要减去 8,以指向新的栈顶位置。
    \item 将操作数的值写入 \mintinline{asm}{%rsp} 指向的地址(Write operand at address given by \mintinline{asm}{%rsp}):最后,将从 \mintinline{asm}{src} 中取出的数据写入到新的栈顶位置,也就是 \mintinline{asm}{%rsp} 所指向的内存地址。
\end{enumerate}

下面是一个简单的入栈操作示例代码:
\begin{minted}{asm}
pushq %rax ; 将寄存器 %rax 中的值入栈
\end{minted}

在这个示例中,首先处理器会读取寄存器 \mintinline{asm}{%rax} 中的值,然后将栈指针 \mintinline{asm}{%rsp} 减 8,最后将 \mintinline{asm}{%rax} 中的值写入到新的栈顶位置。

\subsubsection{x86 - 64 的栈:出栈操作}

出栈操作 \mintinline{asm}{popq dest} 是将栈顶的 64 位数据弹出并写入到目标操作数 \mintinline{asm}{dest} 中。具体步骤如下:

\begin{enumerate}
    \item 从 \mintinline{asm}{%rsp} 指向的地址中取出值(Fetch value in \mintinline{asm}{%rsp}):处理器会从当前栈指针 \mintinline{asm}{%rsp} 所指向的内存地址读取栈顶的数据。
    \item 将值写入 \mintinline{asm}{dest}(Write value to operand \mintinline{asm}{dest}):将读取到的栈顶数据写入到目标操作数 \mintinline{asm}{dest} 中,目标操作数可以是寄存器或者内存地址。
    \item \mintinline{asm}{%rsp} 加 8(Increment \mintinline{asm}{%rsp} by 8):由于已经将栈顶的数据弹出,栈指针需要加 8,以指向新的栈顶位置。

\end{enumerate}

下面是一个简单的出栈操作示例代码:
\begin{minted}{asm}
popq %rbx ; 将栈顶的值弹出并写入寄存器 %rbx
\end{minted}

在这个示例中,处理器会先读取栈顶的值,然后将该值写入到寄存器 \mintinline{asm}{%rbx} 中,最后将栈指针 \mintinline{asm}{%rsp} 加 8。

\subsection{过程调用规范}

\subsubsection{控制流转移}

过程调用是程序中实现代码复用和模块化的重要手段。在 x86 - 64 架构中,过程调用通过 \mintinline{asm}{call label} 指令实现,而过程返回则使用 \mintinline{asm}{ret} 指令。

\mintinline{asm}{call label} 指令的执行过程如下:
\begin{enumerate}
    \item 将返回地址压入栈:返回地址是紧接在 \mintinline{asm}{call} 指令之后的指令所在的地址。当执行 \mintinline{asm}{call} 指令时,处理器会将这个返回地址压入栈中,以便在过程执行完毕后能够返回到正确的位置继续执行。
    \item 跳转至 \mintinline{asm}{label} 标签处的过程代码开始位置:处理器会跳转到 \mintinline{asm}{label} 所指定的地址处,开始执行被调用过程的代码。
\end{enumerate}

\mintinline{asm}{ret} 指令的执行过程如下:

\begin{enumerate}
    \item 从栈中弹出返回地址:处理器会从栈顶弹出之前压入的返回地址。
    \item 跳转至该地址:根据弹出的返回地址,处理器会跳转到相应的位置继续执行程序。
\end{enumerate}

下面是一个完整的过程调用示例代码:
\begin{minted}{asm}
; 示例代码:控制流转移
void multstore(long x, long y, long *dest) {
    long t = mult2(x, y);
    *dest = t;
}
0000000000400540 <multstore>:
    400540: push %rbx        # Save %rbx
    400541: mov %rdx,%rbx    # Save dest
    400544: callq 400550 <mult2> # mult2(x,y)
    400549: mov %rax,(%rbx)  # Save at dest
    40054c: pop %rbx        # Restore %rbx
    40054d: retq            # Return

long mult2(long a, long b) {
    long s = a * b;
    return s;
}
0000000000400550 <mult2>:
    400550: mov %rdi,%rax    # a 
    400553: imul %rsi,%rax   # a * b
    400557: retq            # Return
\end{minted}

在这个示例中,\mintinline{c}{multstore} 函数调用了 \mintinline{c}{mult2} 函数。当执行到 \mintinline{asm}{callq 400550 <mult2>} 指令时,会将下一条指令(\mintinline{asm}{400549})的地址压入栈中,然后跳转到 \mintinline{c}{mult2} 函数的起始地址(\mintinline{asm}{400550})开始执行。当 \mintinline{c}{mult2} 函数执行完毕,执行 \mintinline{asm}{retq} 指令时,会从栈中弹出返回地址(\mintinline{asm}{400549}),然后跳转到该地址继续执行 \mintinline{c}{multstore} 函数的后续代码。

\subsubsection{数据传递}

在 x86 - 64 架构中,数据传递遵循一定的规则。前 6 个参数通过寄存器传递,具体如下:

\begin{table}[H]
    \captionsetup{skip=4pt}
    \centering
    \setlength{\arrayrulewidth}{1pt}
    \begin{tabular}{c}
        \hline
        \makebox[0.8\textwidth][c]{用于传递参数的寄存器} \\
        \noalign{\global\setlength{\arrayrulewidth}{0.5pt}}
        \hline
        \mintinline{asm}{%rdi}:用于传递第一个参数。 \\
        \mintinline{asm}{%rsi}:用于传递第二个参数。 \\
        \mintinline{asm}{%rdx}:用于传递第三个参数。 \\
        \mintinline{asm}{%rcx}:用于传递第四个参数。 \\
        \mintinline{asm}{%r8}:用于传递第五个参数。 \\
        \mintinline{asm}{%r9}:用于传递第六个参数。 \\
        \noalign{\global\setlength{\arrayrulewidth}{1pt}}
        \hline
    \end{tabular}
    \caption{x86 - 64 中传递前 6 个参数的寄存器}
\end{table}

返回值通过 \mintinline{asm}{%rax} 寄存器传递。如果参数数量超过 6 个,剩余的参数则通过栈传递。并且,只有在需要时栈才会分配空间来存储这些额外的参数。

下面是一个参数传递的示例代码:
\begin{minted}{c}
long add_five_numbers(long a, long b, long c, long d, long e) {
    return a + b + c + d + e;
}

int main() {
    long result = add_five_numbers(1, 2, 3, 4, 5);
    return 0;
}
\end{minted}

对应的汇编代码中,参数 \mintinline{c}{1} 会被传递到 \mintinline{asm}{%rdi},参数 \mintinline{c}{2} 会被传递到 \mintinline{asm}{%rsi},参数 \mintinline{c}{3} 会被传递到 \mintinline{asm}{%rdx},参数 \mintinline{c}{4} 会被传递到 \mintinline{asm}{%rcx},参数 \mintinline{c}{5} 会被传递到 \mintinline{asm}{%r8}。函数执行完毕后,返回值会存储在 \mintinline{asm}{%rax} 中。

\subsubsection{存储管理}

在过程执行时,会为该过程分配栈帧空间。栈帧是栈中的一个区域,用于存储过程调用的相关信息,包括参数、局部变量、返回地址等。过程返回前,会释放栈帧空间,以便后续的过程调用使用。

栈帧的管理主要通过栈指针 \mintinline{asm}{%rsp} 和可选的帧指针 \mintinline{asm}{%rbp} 进行。栈指针 \mintinline{asm}{%rsp} 始终指向栈顶,而帧指针 \mintinline{asm}{%rbp} 可以指向栈帧的底部,方便访问栈帧中的数据。

栈帧的内容通常包括以下几个部分:
\begin{table}[H]
    \captionsetup{skip=4pt}
    \centering
    \setlength{\arrayrulewidth}{1pt}
    \begin{tabular}{cc}
        \hline
        \makebox[0.4\textwidth][c]{栈帧内容}    & \makebox[0.4\textwidth][c]{说明}                                                        \\
        \noalign{\global\setlength{\arrayrulewidth}{0.5pt}}
        \hline
        参数(Arguments)                       & 即将要调用的函数的参数或本次调用的参数。如果参数数量超过 6 个,超出部分会存储在栈帧中。                                         \\
        本地变量(Local variables)               & 在寄存器中保存不下的变量。例如,一些数组或者复杂的数据结构通常会存储在栈帧中。                                               \\
        保存的寄存器上下文信息(Saved register context) & 保存过程中使用的寄存器的值,以便恢复。当一个过程需要使用某些寄存器,但又不希望破坏这些寄存器原来的值时,会将这些寄存器的值保存到栈帧中,在过程返回前再恢复。        \\
        指向调用者的栈帧底部的指针(Old frame pointer,可选) & 用于回溯调用栈,可用于访问调用者栈帧中的数据。通过帧指针 \mintinline{asm}{                                        %rbp} 可以方便地找到调用者的栈帧,从而实现对调用者栈帧中数据的访问。 \\
        返回地址(Return address)                & 调用 \mintinline{asm}{call} 指令时入栈,用于过程返回时跳转。当过程执行完毕,需要返回到调用者继续执行时,会从栈帧中取出返回地址,然后跳转到该地址。 \\
        \noalign{\global\setlength{\arrayrulewidth}{1pt}}
        \hline
    \end{tabular}
    \caption{x86 - 64/Linux 栈帧内容}
\end{table}

\subsubsection{寄存器使用惯例}

在 x86 - 64 架构中,寄存器分为“调用者保护”和“被调用者保护”两类,这是为了保证在过程调用过程中寄存器的值不会被意外修改。

“调用者保护”的寄存器,调用者在调用前把临时数据保存到自己的栈帧中。这意味着,如果一个过程(调用者)需要使用这些寄存器,并且希望在调用其他过程后这些寄存器的值仍然保持不变,那么它需要在调用之前将这些寄存器的值保存到自己的栈帧中,在调用返回后再恢复这些值。

“被调用者保护”的寄存器,被调用者在使用前将临时数据保存至自己的栈帧中,并在返回前恢复这些数据。也就是说,当一个过程(被调用者)需要使用这些寄存器时,它需要先将这些寄存器的值保存到自己的栈帧中,在使用完后再恢复这些值,以确保调用者看到的这些寄存器的值没有被改变。

具体的寄存器分类如下:
\begin{table}[H]
    \captionsetup{skip=4pt}
    \centering
    \setlength{\arrayrulewidth}{1pt}
    \begin{tabular}{cc}
        \hline
        \makebox[0.4\textwidth][c]{寄存器类别} & \makebox[0.4\textwidth][c]{寄存器列表} \\
        \noalign{\global\setlength{\arrayrulewidth}{0.5pt}}
        \hline
        调用者保护(Caller - saved)             & \mintinline{asm}{                 %rax}, \mintinline{asm}{%rdi}, \mintinline{asm}{%rsi}, \mintinline{asm}{%rdx}, \mintinline{asm}{%rcx}, \mintinline{asm}{%r8}, \mintinline{asm}{%r9}, \mintinline{asm}{%r10}, \mintinline{asm}{%r11} \\
        被调用者保护(Callee - saved)            & \mintinline{asm}{                 %rbx}, \mintinline{asm}{%r12}, \mintinline{asm}{%r13}, \mintinline{asm}{%r14}, \mintinline{asm}{%rbp}, \mintinline{asm}{%rsp} \\
        \noalign{\global\setlength{\arrayrulewidth}{1pt}}
        \hline
    \end{tabular}
    \caption{x86 - 64 Linux 中寄存器的保护类别}
\end{table}

下面是一个寄存器使用惯例的示例代码:
\begin{minted}{c}
void function1() {
    long temp = 10;
    function2();
    // 此时 temp 的值仍然是 10,因为 %rax 是调用者保护寄存器
}

void function2() {
    long result = 20;
    // 可以安全地使用 %rax 寄存器,因为调用者已经保存了 %rax 的值
    // ...
}
\end{minted}

在这个示例中,\mintinline{c}{function1} 是调用者,\mintinline{c}{function2} 是被调用者。如果 \mintinline{c}{function1} 需要使用 \mintinline{asm}{%rax} 寄存器,并且希望在调用 \mintinline{c}{function2} 后 \mintinline{asm}{%rax} 的值不变,那么它需要在调用之前保存 \mintinline{asm}{%rax} 的值。而 \mintinline{c}{function2} 可以安全地使用 \mintinline{asm}{%rax} 寄存器,因为调用者已经处理了 \mintinline{asm}{%rax} 的保存和恢复。

\subsection{递归}

递归是一种函数调用自身的编程技术,它在解决一些复杂问题时非常有用。下面以计算无符号长整型数中 1 的个数的递归函数 \mintinline{c}{pcount_r} 为例进行说明。

\subsubsection{递归函数的代码实现}

\begin{minted}{c}
/* Recursive popcount */
long pcount_r(unsigned long x) {
    if (x == 0)
        return 0;
    else
        return (x & 1) + pcount_r(x >> 1);
}
\end{minted}

这个函数的功能是计算无符号长整型数 \mintinline{c}{x} 中二进制表示下 1 的个数。如果 \mintinline{c}{x} 为 0,则返回 0;否则,返回 \mintinline{c}{x} 的最低位(\mintinline{c}{x & 1})加上 \mintinline{c}{x} 右移一位后剩余部分中 1 的个数(通过递归调用 \mintinline{c}{pcount_r(x >> 1)} 计算)。

\subsubsection{递归函数的汇编代码分析}

对应的汇编代码如下:
\begin{minted}{asm}
pcount_r:
    movl $0, %eax
    testq %rdi, %rdi
    je .L6
    pushq %rbx
    movq %rdi, %rbx
    andl $1, %ebx
    shrq %rdi # (by 1)
    call pcount_r
    addq %rbx, %rax
    popq %rbx
.L6:
    rep; ret
\end{minted}

下面对这段汇编代码进行详细分析:
\begin{enumerate}
    \item \mintinline{asm}{movl $0, %eax}:将寄存器 \mintinline{asm}{%eax} 初始化为 0,用于存储最终的结果。
    \item \mintinline{asm}{testq %rdi, %rdi}:对寄存器 \mintinline{asm}{%rdi} 进行按位与操作,用于判断 \mintinline{asm}{%rdi} 是否为 0。如果为 0,则设置相应的标志位。
    \item \mintinline{asm}{je .L6}:如果 \mintinline{asm}{%rdi} 为 0(即 \mintinline{c}{x} 为 0),则跳转到标签 \mintinline{asm}{.L6} 处,直接返回 0。
    \item \mintinline{asm}{pushq %rbx}:将寄存器 \mintinline{asm}{%rbx} 的值压入栈中,因为 \mintinline{asm}{%rbx} 是被调用者保护寄存器,需要在使用前保存其值。
    \item \mintinline{asm}{movq %rdi, %rbx}:将参数 \mintinline{c}{x}(存储在 \mintinline{asm}{%rdi} 中)复制到寄存器 \mintinline{asm}{%rbx} 中。
    \item \mintinline{asm}{andl $1, %ebx}:对 \mintinline{asm}{%ebx} 的最低位进行按位与操作,得到 \mintinline{c}{x} 的最低位。
    \item \mintinline{asm}{shrq %rdi}:将 \mintinline{asm}{%rdi} 中的值右移一位,相当于 \mintinline{c}{x >> 1}。
    \item \mintinline{asm}{call pcount_r}:递归调用 \mintinline{c}{pcount_r} 函数,计算 \mintinline{c}{x >> 1} 中 1 的个数。
    \item \mintinline{asm}{addq %rbx, %rax}:将 \mintinline{c}{x} 的最低位(存储在 \mintinline{asm}{%rbx} 中)加到结果寄存器 \mintinline{asm}{%rax} 中。
    \item \mintinline{asm}{popq %rbx}:从栈中弹出之前保存的 \mintinline{asm}{%rbx} 的值,恢复 \mintinline{asm}{%rbx} 的原始值。
    \item \mintinline{asm}{rep; ret}:返回调用者,将结果存储在 \mintinline{asm}{%rax} 中。
\end{enumerate}

递归函数的处理和普通函数调用类似,栈帧为每次函数调用提供私有的存储空间,用于保存寄存器和局部变量、返回地址等。寄存器使用惯例保证了一次函数调用不会破坏其他函数的数据,栈的后进先出特性也适用于递归和相互递归的情况。
